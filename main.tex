\PassOptionsToPackage{unicode=true}{hyperref} % options for packages loaded elsewhere
\PassOptionsToPackage{hyphens}{url}
\PassOptionsToPackage{dvipsnames,svgnames*,x11names*,table}{xcolor}
%
%\documentclass[a4paper,]{scrartcl}
\documentclass{article}

\usepackage{amssymb,amsmath}
\usepackage{ifxetex,ifluatex}
\usepackage{fixltx2e} % provides \textsubscript
\ifnum 0\ifxetex 1\fi\ifluatex 1\fi=0 % if pdftex
  \usepackage[T1]{fontenc}
  \usepackage[utf8]{inputenc}
  \usepackage{textcomp} % provides euro and other symbols
\else % if luatex or xelatex
  \usepackage{unicode-math}
  \defaultfontfeatures{Ligatures=TeX,Scale=MatchLowercase}
\fi
% use upquote if available, for straight quotes in verbatim environments
\IfFileExists{upquote.sty}{\usepackage{upquote}}{}

% use microtype if available
\IfFileExists{microtype.sty}{
\usepackage[]{microtype}
\UseMicrotypeSet[protrusion]{basicmath} % disable protrusion for tt fonts
}{}

\IfFileExists{parskip.sty}{%
\usepackage{parskip}
}{% else
\setlength{\parindent}{0pt}
\setlength{\parskip}{6pt plus 2pt minus 1pt}
}

\usepackage{xcolor}
\definecolor{default-citecolor}{HTML}{4077C0}
\definecolor{default-urlcolor}{HTML}{4077C0}

\usepackage{hyperref}
\hypersetup{
            pdftitle={The Raft Consensus Algorithm},
            pdfauthor={Fabio Anderegg},
            colorlinks=true,
            linkcolor=Maroon,
            citecolor=default-citecolor,
            urlcolor=cyan,
            breaklinks=true}
\urlstyle{same}  % don't use monospace font for urls
\usepackage[margin=2.5cm,includehead=true,includefoot=true,centering]{geometry}
\setlength{\emergencystretch}{3em}  % prevent overfull lines
\providecommand{\tightlist}{%
  \setlength{\itemsep}{0pt}\setlength{\parskip}{0pt}}
\setcounter{secnumdepth}{5}
% Redefines (sub)paragraphs to behave more like sections
\ifx\paragraph\undefined\else
\let\oldparagraph\paragraph
\renewcommand{\paragraph}[1]{\oldparagraph{#1}\mbox{}}
\fi
\ifx\subparagraph\undefined\else
\let\oldsubparagraph\subparagraph
\renewcommand{\subparagraph}[1]{\oldsubparagraph{#1}\mbox{}}
\fi

% Make use of float-package and set default placement for figures to H
\usepackage{float}
\floatplacement{figure}{H}


\title{The Raft Consensus Algorithm}
\author{Fabio Anderegg}
\date{2017-12-04}





%%
%% added
%%

%
% No language specified? take American English.
%

\ifnum 0\ifxetex 1\fi\ifluatex 1\fi=0 % if pdftex
  \usepackage[shorthands=off,main=english]{babel}
\else
  % See issue https://github.com/reutenauer/polyglossia/issues/127
  \renewcommand*\familydefault{\sfdefault}
  % load polyglossia as late as possible as it *could* call bidi if RTL lang (e.g. Hebrew or Arabic)
  \usepackage{polyglossia}
  \setmainlanguage[]{english}
\fi


%
% colors
%

%
% for the background color of the title page
%
\usepackage{pagecolor}
\usepackage{afterpage}

%
% TOC depth and 
% section numbering depth
%
\setcounter{tocdepth}{3}
\setcounter{secnumdepth}{3}

%
% line spacing
%
\usepackage{setspace}
\setstretch{1.2}

%
% break urls
%
\PassOptionsToPackage{hyphens}{url}

%
% When using babel or polyglossia with biblatex, loading csquotes is recommended 
% to ensure that quoted texts are typeset according to the rules of your main language.
%
\usepackage{csquotes}

%
% captions
%
\usepackage[font={small,it}]{caption}
\newcommand{\imglabel}[1]{\textbf{\textit{(#1)}}}

%
% blockquote
%
\definecolor{blockquote-border}{RGB}{221,221,221}
\definecolor{blockquote-text}{RGB}{119,119,119}
\usepackage{mdframed}
\newmdenv[rightline=false,bottomline=false,topline=false,linewidth=3pt,linecolor=blockquote-border,skipabove=\parskip]{customblockquote}
\renewenvironment{quote}{\begin{customblockquote}\list{}{\rightmargin=0em\leftmargin=0em}%
\item\relax\color{blockquote-text}\ignorespaces}{\unskip\unskip\endlist\end{customblockquote}}

%
% Source Sans Pro as the default font family
% Source Code Pro for monospace text
%
% 'default' option sets the default 
% font family to Source Sans Pro, not \sfdefault.
%
\usepackage[default]{sourcesanspro}
\usepackage{sourcecodepro}

%
% heading color
%
\definecolor{heading-color}{RGB}{40,40,40}
%\addtokomafont{section}{\color{heading-color}}
% When using the classes report, scrreprt, book, 
% scrbook or memoir, uncomment the following line.
%\addtokomafont{chapter}{\color{heading-color}}

%
% variables for title and author
%
\usepackage{titling}
\title{The Raft Consensus Algorithm}
\author{Fabio Anderegg}

%
% environment for boxes
%
%\usepackage{framed}

%
% tables
%

%
% remove paragraph indention
%
\setlength{\parindent}{0pt}
\setlength{\parskip}{6pt plus 2pt minus 1pt}
\setlength{\emergencystretch}{3em}  % prevent overfull lines

%
% header and footer
%
\usepackage{fancyhdr}
\pagestyle{fancy}
\fancyhead{}
\fancyfoot{}
\lhead{The Raft Consensus Algorithm}
\chead{}
\rhead{2017-12-04}
\lfoot{Fabio Anderegg}
\cfoot{}
\rfoot{\thepage}
\renewcommand{\headrulewidth}{0.4pt}
\renewcommand{\footrulewidth}{0.4pt}

\usepackage{graphicx}
\graphicspath{ {images/} {scenarios/election/} }

\usepackage{enumitem}
\setitemize{noitemsep,topsep=0pt,parsep=0pt,partopsep=0pt}


\usepackage{tabularx}
\usepackage{calc}


\begin{document}

%%
%% begin titlepage
%%

\begin{titlepage}
\newgeometry{left=6cm}
\definecolor{titlepage-color}{HTML}{2196F3}
\newpagecolor{titlepage-color}\afterpage{\restorepagecolor}
\newcommand{\colorRule}[3][black]{\textcolor[HTML]{#1}{\rule{#2}{#3}}}
\begin{flushleft}
\noindent
\\[-1em]
\color[HTML]{FFFFFF}
\makebox[0pt][l]{\colorRule[1976D2]{1.3\textwidth}{8pt}}
\par
\noindent

{ \setstretch{1.4}
\vfill
\noindent {\huge \textbf{\textsf{The Raft Consensus Algorithm}}}
\vskip 2em
\noindent
{\Large \textsf{\MakeUppercase{Fabio Anderegg}}
\vfill
}

\textsf{2017-12-04}}
\end{flushleft}
\end{titlepage}
\restoregeometry

%%
%% end titlepage
%%


{
\hypersetup{linkcolor=}
\setcounter{tocdepth}{2}
\tableofcontents
\pagebreak
}


















\section{Overview}\label{overview}

\includegraphics[scale=0.1]{raft}

This paper cover Raft, a consensus algorithm for distributed systems.

Raft wants to be a simpler replacement for PAXOS, an algorithm developed by Leslie Lamport and considered the reference consensus algorithm for distributed systems[CN]. PAXOS suffers from two big problems, which Raft tries to solve: It is very hard to understand [CN] and it has some missing pieces to actually implement a real world system using PAXOS [CN (Chubby?)].

Raft was developed by Diego Ongaro and extensivly described in this PhD thesis [CN]. He not only describes how the algorithm works, bot also describes a study under students to test if the algorithm is actually easier to understand than PAXOS. Raft has been formally verified to be correct [CN] and implemented in multiple real world systems, e.g. in etcd [CN], Consul [CN] and RethinkDB [CN].

The offical GitHub Page [https://raft.github.io/] contains many further references about the algorithm, a simulator for the algorithm from which screenshots have been used in this paper and an extensive list of implementations in many different programming languages.

This paper has four parts: The first part explains what consensus in a distributed system is and how it is used, the second part explains how the Raft algorithm works, the third part shows how Raft handles some failure modes. To validate the claim that Raft is easy to implement, the author of this paper tried to implement a simple key/value store using Raft, which is described in the fourth part.

\section{Technologies}\label{technologies}

\subsection{Docker}\label{docker}

Docker is a container runtime for Linux. It is an operating system level
virtualization technology, which allows to run software isolated from
each other. Docker provides the runtime to run a container on a single
machine and a mechanism to generate images which can be run on Docker
\footnote{\url{https://docs.docker.com/engine/docker-overview/}}.

\subsection{Kubernetes}\label{kubernetes}

Kubernetes is a container orchestration software developed by Google. A
container orchestration software is required to deploy and run
containers on multiple hosts and to allow these containers to exchange
data over the network. Kubernetes can be run standalone or inside a
bigger software package like OpenShift or Cloud Foundry. There are also
hosted Kubernetes solutions like Google Kubernetes Engine (GKE), Azure
Container Service or OpenShift dedicated\footnote{\url{https://kubernetes.io/docs/setup/pick-right-solution/\#hosted-solutions}}.

\end{document}
